\frame
{
    \frametitle{D {\Dname} {by \itshape JohnVictor}}

    回顾题意: 有一种填了 $\mathbb Z_3$ 颜色的 $n\times m$ 网格.

    称一次演化为: 对于一个颜色为 $c$ 的格子, 如果有颜色为 $c-1$ 的相邻格子,
    那么在下一轮这个格子的颜色变成 $c-1$, 否则不变.

    称一个染色方案是 \textbf{好的}, 当且仅当在有限次演化之后所有格子颜色相同.

    对于一个好的染色方案, 称它的权值为最小的正整数 $t$, 满足网格演化 $t$ 次后, 左上角
    格子颜色再也不变.

    网格上有些格子已经确定颜色, 有些没有, 求好的染色方案数, 以及它们的权值和.

    $2\leq n\leq 5, 2\leq m\leq 50$.
}

\frame
{
    \frametitle{题解}

    注意到
    \[ \begin{bmatrix}
        0 & 1\\
        2 & 2
    \end{bmatrix} \mapsto \begin{bmatrix}
        2 & 0\\
        1/2 & 1
    \end{bmatrix}, \]
    可知形如这种矩阵 (或其翻转旋转, 整体加一个数) 一定会转移到自己.

    \pause

    我们称这种结构 \textbf{阻塞} 了一个网格成为好的.

    \pause

    \textbf{断言}: 不存在阻塞的网格都是好的.
}

\frame
{
    \frametitle{题解}

    现在已知网格 $C\in \mathbb Z_3^{n\times m}$, 它的所有 $2\times 2$ 子网格, 都不是阻塞结构.

    我们一定可以找到一种给每个格子赋予一个 {\textbf{整数}} 的方案 $F\in \mathbb Z^{n\times m}$,
    使得:
    \begin{itemize}
        \item $C_{ij} = F_{ij} \bmod 3$.
        \item $F$ 的相邻两项差不超过 $1$.
    \end{itemize}
}

\frame
{
    \frametitle{题解}

    \[ \Huge {C = \begin{bmatrix}
        0 & 0 & 2\\
        1 & 0 & 0\\
        1 & 1 & 0
    \end{bmatrix}, F = \begin{bmatrix}
        0 & \onslide<2->{0} & \onslide<3->{-1} \\
        \onslide<4->{1} & \onslide<4->{0} & \onslide<4->{0} \\
        \onslide<5->{1} & \onslide<5->{1} & \onslide<5->{0}
    \end{bmatrix}} \]

    \onslide<3->{$F_{ij} \in F_{i(j-1)} + \{-1, 0, 1\}$ 保证 $F$ 被唯一确定.}

    \onslide<5->{没有阻塞结构, 保证 $F$ 合法 (保证不用于生成 $F$ 的那部分约束被满足).}
}


\frame
{
    \frametitle{题解}

    \only<1>{
        \[ \Huge {C = \begin{bmatrix}
            0 & 0 & 2\\
            1 & 0 & 0\\
            1 & 1 & 0
        \end{bmatrix}, F = \begin{bmatrix}
            0 & 0 & -1\\
            1 & 0 & 0\\
            1 & 1 & 0
        \end{bmatrix}} \]
    }

    \only<2>{
        \[ \Huge {C = \begin{bmatrix}
            0 & 2 & 2\\
            0 & 0 & 2\\
            1 & 0 & 0
        \end{bmatrix}, F = \begin{bmatrix}
            0 & -1 & -1\\
            0 & 0 & -1\\
            1 & 0 & 0
        \end{bmatrix}} \]
    }

    \only<3>{
        \[ \Huge {C = \begin{bmatrix}
            2 & 2 & 2\\
            0 & 2 & 2\\
            0 & 0 & 2
        \end{bmatrix}, F = \begin{bmatrix}
            -1 & -1 & -1\\
            0 & -1 & -1\\
            0 & 0 & -1
        \end{bmatrix}} \]
    }

    \only<4>{
        \[ \Huge {C = \begin{bmatrix}
            2 & 2 & 2\\
            2 & 2 & 2\\
            0 & 2 & 2
        \end{bmatrix}, F = \begin{bmatrix}
            -1 & -1 & -1\\
            -1 & -1 & -1\\
            0 & -1 & -1
        \end{bmatrix}} \]
    }
    \only<5->{
        \[ \Huge {C = \begin{bmatrix}
            2 & 2 & 2\\
            2 & 2 & 2\\
            2 & 2 & 2
        \end{bmatrix}, F = \begin{bmatrix}
            -1 & -1 & -1\\
            -1 & -1 & -1\\
            -1 & -1 & -1
        \end{bmatrix}} \]
    }

    \begin{itemize}
        \item<5-> $C$ 上的演化可以对应于 $F$ 上的演化.
        \item<5-> $F$ 上的演化: 每次把格子变成自己和周围的 $\max$.
        \item<6-> 稳定态: 所有数稳定到 $F$ 的\textbf{最小值}.
        \item<7-> 权值: 左上角格子到 $F$ 的一个最小值的\textbf{最短路}.
    \end{itemize}
}


\frame
{
    \frametitle{题解}
    
    不妨设 $n\leq m$, 按照列从小到大, 按照顺序一个个填写颜色, 维护轮廓线上 $F$ 的值.

    \onslide<2->{
    轮廓上的 $F$ 本身的状态, 只需要知道相邻数的差, 状态数为 $O(3^n)$.
    }

    \onslide<3->{
        \only<3>{但我们还需要知道左上角距离目前看到的最小值的最短距离, 这个数大小 $O(m)$.}
        \only<4>{\sout{但我们还需要知道左上角距离目前看到的最小值的最短距离, 这个数大小 $O(m)$.}}

        还需要知道目前轮廓线上最小的数比最小值大多少, 这个数大小 $O(m)$.

        \only<3>{时间复杂度 $O(nm \cdot 3^n m^2)$.}
        \only<4>{\sout{时间复杂度 $O(nm \cdot 3^n m^2)$.}}
    }

    \onslide<4->{在更新最小值的时候决策这是不是最后的最小值, 如果是, 那么最短距离最多再多 $n$, 可以据此优化状态,
    做到时间复杂度 $O(nm \cdot 3^n nm)$.}
}
