\frame
{
	\frametitle{I 勿蹖宠物 {by \itshape SpiritualKhorosho}}

	给定 $s_1, \cdots, s_N$,求满足以下条件的下标序列 $i_1, i_2, \cdots, i_k$($k$ 为任意正整数)的数量,其中 $+$ 表示字符串的连接:
	\begin{itemize}
		\item $s_{i_1} + s_{i_2} + \cdots + s_{i_k}$ 是长度为 $L$ 的回文串。
	\end{itemize}

	保证 $1\le N\le 333$,$1\le L\le 1000$,$\sum_{i=1}^N \left|s_i\right| \le 600$。

	\begin{block}{题目名称是什么意思?}
		使用样例 3 给出的单词,可以还原出一个有名的英文回文句子,而题目名称是这个句子的中文翻译。除此之外,题目名称的第一个字和第四个字、第二个字和第三个字的拼音(不考虑声调)是相同的。
	\end{block}

}

\begin{frame}{分析}
	
	在做字符串匹配相关的题目时,有一类比较常见的策略是记当前匹配到了哪一位,并且这一位对应自动机中的哪个结点。

	本题需要满足整个串是回文串,因此不好单向维护匹配情况。为了解决这个问题,我们不妨从两侧往内同时匹配:记 $f(i,\cdots)$ 表示当前处理到第 $i$ 位及第 $(L-i+1)$ 位,匹配情况为 $\cdots$ 时对应的方案数。\pause

	通过 $f(i,\cdots)$ 转移到 $f(i+1,\cdots)$ 时,需要枚举下一个字母 $\alpha$,让第 $i$ 位对应的指向自动机 $q_l$ 结点的指针找对应 $\alpha$ 的正向转移的边,第 $(L-i+1)$ 位对应的指向自动机 $q_r$ 结点的指针找对应 $\alpha$ 的反向转移的边。正向转移可以使用 Trie 树简单维护,但是反向转移比较麻烦。\pause

	注意到可以反向建立 Trie 树,这样就方便反向找了。

\end{frame}

\begin{frame}{正向 Trie 及反向 Trie}
	
	\begin{center}
		\tikzstyle{vertex} = [circle, inner sep = 0pt, minimum size = 0.5cm, draw, fill = white, thick]
		\tikzstyle{imagine} = [vertex, dashed]
		\begin{tikzpicture}[every edge quotes/.style={pos = 0.35, fill = white, inner sep = 1pt, font = \footnotesize}, shorten >= 1pt, >=stealth']
			\foreach \name/\x/\y/\d in {P0/0/0/, P1/1/1/, P2/1/-1/, P3/1/-2/, P4/2/2/, P5/2/0.5/, P6/2/-1/, P7/2/-2/, P8/3/2/double, P9/3/1/double, P10/3/0/, P11/3/-1/double, P12/3/-2/, P13/4/0/double, P14/4/-2/, P15/5/-2/double} {
				\path node[vertex, \d] at (\x, \y) (\name) {};
			}
			\foreach \u/\v/\a in {P0/P1/e, P0/P2/l, P0/P3/o, P1/P4/e, P1/P5/v, P2/P6/e, P3/P7/l, P4/P8/l, P5/P9/e, P5/P10/i, P6/P11/e, P7/P12/i, P10/P13/l, P12/P14/v, P14/P15/e} {
				\draw [->] (\u) edge ["\a"] (\v);
			}

			\foreach \name/\x/\y/\d in {Q0/10/0/, Q1/9/1/, Q2/9/-1.5/, Q3/8/2/, Q4/8/0.5/, Q5/8/-1/, Q6/8/-2/, Q7/7/2/double, Q8/7/1/double, Q9/7/0/, Q10/7/-1/double, Q11/7/-2/, Q12/6/0/, Q13/6/-2/double, Q14/5/0/double} {
				\path node[vertex, \d] at (\x, \y) (\name) {};
			}
			\foreach \u/\v/\a in {Q0/Q1/e, Q0/Q2/l, Q1/Q3/e, Q1/Q4/v, Q2/Q5/e, Q2/Q6/i, Q3/Q7/l, Q4/Q8/e, Q4/Q9/i, Q5/Q10/e, Q6/Q11/v, Q9/Q12/l, Q11/Q13/e, Q12/Q14/o} {
				\draw [->] (\u) edge ["\a"] (\v);
			}
			\draw[violet, dashed] (4.5, 3) -- (4.5, -0.25) -- (5.5, -1.75) -- (5.5, -3);
		\end{tikzpicture}
	\end{center}

\end{frame}


\begin{frame}{做法}
	
	记 $f\left(i,q_l, q_r\right)$ 表示当前处理到第 $i$ 位及第 $(L-i+1)$ 位,第 $i$ 位对应正向 Trie 树上的结点 $q_l$,第 $(L-i+1)$ 位对应反向 Trie 树上的结点 $q_r$。转移时枚举第 $(i+1)$ 位及第 $(L-i)$ 位的字母 $\alpha$,如果 $q_l$ 和 $q_r$ 均有 $\alpha$ 的转移边,则相应转移给 $f\left(i+1,q'_l,q'_r\right)$。

	需要注意在 $i=L/2$ 处特判 $q_l$ 和 $q_r$ 能否合并:如果是奇数,则两个结点需要能通过同一个字母走到同一个单词的相同位置;如果是偶数,则有可能两边都恰好走到单词末尾,也有可能为同一个单词的相邻位置。

	\begin{itemize}
		\item stac\textbf{k}/cats
		\item step/o\textbf{n}/\textbf{n}o/pets
		\item n\textbf{oo}n
	\end{itemize}

\end{frame}

\begin{frame}{做法}
	
	标程的处理方法是记 $f\left(i,q_l, q_r\right)$ 表示已知两侧分别匹配到 $q_l$ 及 $q_r$ 时,第 $i$ 至 $(L-i+1)$ 位的合法方案数。可以先根据单词表计算 $f\left(\left\lceil L/2\right\rceil, \cdot, \cdot\right)=0/1$,再根据 $f(i+1,\cdot,\cdot)$ 往外扩展计算 $f(i,\cdot,\cdot)$。直接用循环计算 DP 可以滚动数组,不用担心被卡空间。
	\begin{itemize}
		\item 当然,本题的空间限制最后开到了 1024MiB,只要不开两个 $500\times 600^2$ 的 int 数组都不会被卡。如果你想写记忆化搜索,用 char 记每个状态是否被访问过也是可以通过本题的。
	\end{itemize}

	复杂度看上去是 $O\left(L \left(\sum_{i=1}^N \left|s_i\right|\right)^2 |\Sigma|\right)$,但是实际上状态不会很满。比较优秀的写法是枚举 $q_l$ 及 $q_r$ 转移边,这样可以做到严格 $O\left(L \left(\sum_{i=1}^N \left|s_i\right|\right)^2\right)$,可以轻松通过本题。

\end{frame}

\begin{frame}{$N$ 的范围?}
	
	$$1\times 26 + 2 \times (333 - 26) = 640 > 600.$$

	$$\frac{600-1\times 26}{2} = 287.$$

\end{frame}
