\frame
{
  \frametitle{K 报数 IV {by \itshape liuzhangfeiabc}}

	设 $f(n)$ 为 $n$ 的十进制各位数字之和.
	设 $f_k(n)=f(f(\cdots f(n)))$(复合 $k$ 次).
	给定正整数 $N, k, m$,求有多少 $n\in[1,N]$ 满足 $f_k(n)=m$.

	$N \le 10^{1000}, k, m \le 10^9$.

}

\begin{frame}{题解}
	不超过 $10^{1000}$ 的数中,$f(n)$ 最大为 $f(999\dots 9)$( $1000$ 个 $9$)$=9000$.

	不超过 $9000$ 的数中,$f(n)$ 最大为 $f(8999)=35$.

	不超过 $35$ 的数中,$f(n)$ 最大为 $f(29)=11$.

	不超过 $11$ 的数中,$f(n)$ 最大为 $9$.

	因此,任何不超过 $10^{1000}$ 的数最多作用 $4$ 次 $f$ 就会变成一位数,以后就再也不会变了.

	因此 $m$ 和 $k$ 的数据范围是唬人的.
\end{frame}

\begin{frame}{题解}

	对于原问题,我们可以把作用 $k$ 次 $f$ 的过程拆成两部分:

	\begin{itemize}
	\item 先作用一次 $f$,得到一个不超过 $9000$ 的数.
	\item 再作用 $k-1$ 次 $f$ 得到 $m$.
	\end{itemize}

	我们直接预处理出 $1\dots 9000$ 的所有数进行 $0\dots 4$ 次 $f$ 作用之后变成什么.

	然后对每个 $1\dots 9000$ 中的 $x$,计算 $1\dots N$ 中有多少个数的数位之和是 $x$.

	\begin{itemize}
	\item 直接大力数位 dp: $dp(i,x,0/1)$ 表示前 $i$ 位,数位之和为 $x$,是否卡上界的数的个数即可.
	\end{itemize}

	然后对每个 $1\dots 9000$ 中的 $x$ 检查一下:$x$ 作用 $k-1$ 次 $f$ 之后是否等于 $m$,是的话就加上 $dp(x)$ 的贡献.

\end{frame}